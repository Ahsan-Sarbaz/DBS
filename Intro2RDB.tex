\PassOptionsToPackage{unicode=true}{hyperref} % options for packages loaded elsewhere
\PassOptionsToPackage{hyphens}{url}
%
\documentclass[]{article}
\usepackage{lmodern}
\usepackage{amssymb,amsmath}
\usepackage{ifxetex,ifluatex}
\usepackage{fixltx2e} % provides \textsubscript
\ifnum 0\ifxetex 1\fi\ifluatex 1\fi=0 % if pdftex
  \usepackage[T1]{fontenc}
  \usepackage[utf8]{inputenc}
  \usepackage{textcomp} % provides euro and other symbols
\else % if luatex or xelatex
  \usepackage{unicode-math}
  \defaultfontfeatures{Ligatures=TeX,Scale=MatchLowercase}
\fi
% use upquote if available, for straight quotes in verbatim environments
\IfFileExists{upquote.sty}{\usepackage{upquote}}{}
% use microtype if available
\IfFileExists{microtype.sty}{%
\usepackage[]{microtype}
\UseMicrotypeSet[protrusion]{basicmath} % disable protrusion for tt fonts
}{}
\IfFileExists{parskip.sty}{%
\usepackage{parskip}
}{% else
\setlength{\parindent}{0pt}
\setlength{\parskip}{6pt plus 2pt minus 1pt}
}
\usepackage{hyperref}
\hypersetup{
            pdfborder={0 0 0},
            breaklinks=true}
\urlstyle{same}  % don't use monospace font for urls
\usepackage{listings}
\newcommand{\passthrough}[1]{#1}
\usepackage{longtable,booktabs}
% Fix footnotes in tables (requires footnote package)
\IfFileExists{footnote.sty}{\usepackage{footnote}\makesavenoteenv{longtable}}{}
\setlength{\emergencystretch}{3em}  % prevent overfull lines
\providecommand{\tightlist}{%
  \setlength{\itemsep}{0pt}\setlength{\parskip}{0pt}}
\setcounter{secnumdepth}{0}
% Redefines (sub)paragraphs to behave more like sections
\ifx\paragraph\undefined\else
\let\oldparagraph\paragraph
\renewcommand{\paragraph}[1]{\oldparagraph{#1}\mbox{}}
\fi
\ifx\subparagraph\undefined\else
\let\oldsubparagraph\subparagraph
\renewcommand{\subparagraph}[1]{\oldsubparagraph{#1}\mbox{}}
\fi

% set default figure placement to htbp
\makeatletter
\def\fps@figure{htbp}
\makeatother


\date{}

\begin{document}

\textbf{Contents presented here have been taken from 2nd Chapter
Introduction to Relational Databases of the book Database Systems An
Application-Oriented Approach 2nd Edition by Michael Kifer, Arthur
Bernstein and Philip M. Lewis }

\hypertarget{introduction-to-relational-databases}{%
\section{Introduction to Relational
Databases}\label{introduction-to-relational-databases}}

\hypertarget{table-rows-and-columns}{%
\subsection{Table, Rows and columns}\label{table-rows-and-columns}}

In Relational databases data is stored in tables.

For example, Student Registration System might include the STUDENT
table.

\begin{longtable}[]{@{}llll@{}}
\caption{The table STUDENT. Each row describes a single
student.}\tabularnewline
\toprule
Id & Name & Address & Status\tabularnewline
\midrule
\endfirsthead
\toprule
Id & Name & Address & Status\tabularnewline
\midrule
\endhead
1111111 & John Doe & 423 Main St. & Freshman\tabularnewline
666666666 & Joseph Public & 666 Hollow Rd. & Sophomore\tabularnewline
111223344 & Mary Smith & 1 Lake St. & Freshman\tabularnewline
987654321 & Bart Simpson & Fox 5 TV & Senior\tabularnewline
023456789 & Homer Simpson & Fox 5 TV & Senior\tabularnewline
123454321 & Joe Blow & 6 Yard Ct. & Junior\tabularnewline
& & &\tabularnewline
\bottomrule
\end{longtable}

\begin{itemize}
\tightlist
\item
  A table contains a set of rows. Each row contains information about
  one student.
\item
  Each column of the table describes the student in a particular way.
\item
  In example above the columns are id, Name, Address, and Status.
\item
  Each column has an associated type, called its domain, from which the
  value in a particular row for that column is drawn. For example the
  domain for Id is integer and the domain for Name is string.
\end{itemize}

This database model is called ``relational'' because it is based on the
mathematical concept of relation.

A \textbf{mathematical relation} captures the notion that elements of
different sets are related to one another.

For example, \emph{John Doe}, an element of the set of all humans, is
related to \emph{123 Main St.}, an element of the set of all addresses,
and to \emph{1111111111,} an element of the set of all Ids.

A relation is a set of \textbf{tuples}.

For example, of the table \textbf{STUDENT}, we might define a relation
called \textbf{STUDENT} containing the tuple (1111111, John Doe, 423
Main St., Freshman).

\begin{longtable}[]{@{}ll@{}}
\caption{Correspondence between tables and relations}\tabularnewline
\toprule
Tables & Relations\tabularnewline
\midrule
\endfirsthead
\toprule
Tables & Relations\tabularnewline
\midrule
\endhead
Rows & Tuples\tabularnewline
Columns & Attributes\tabularnewline
\bottomrule
\end{longtable}

\hypertarget{operations-on-tables-are-mathematically-defined}{%
\subsection{Operations on tables are mathematically
defined}\label{operations-on-tables-are-mathematically-defined}}

In most applications, the database is under the control of a database
management systems (DBMS). When an application wants to perform an
operation on the database, it does so by making a request to the DBMS.

A typical operation might

\begin{itemize}
\tightlist
\item
  extract some information from the rows of one or more tables,
\item
  add rows, or
\item
  delete rows.
\end{itemize}

In addition to the fact that tables in the database can be modeled by
mathematical relations, operations on the tables can also be modeled as
mathematical operations on the corresponding relations.

Thus, a particular unary operation might take a table, T, as an argument
and produce a result table containing a subset of the rows of T.

For example:

\begin{enumerate}
\def\labelenumi{\arabic{enumi}.}
\tightlist
\item
  an instructor might want to display the roster of students registered
  for a course. Such a request might involve scanning the TRANSCRIPT
  table, locating the rows corresponding to the course, and returning
  them to the application.\\
\item
  a particular binary operation might take two tables as arguments and
  construct a new table containing the union of the rows of the argument
  tables.
\end{enumerate}

A complex query against a database might be equivalent to an expression
involving many such relational operations involving many tables.

Because of this mathematical description, relational operations can be
precisely defined and their mathematical properties, such as
commutativity and associativity, can be proven.

This mathematical description has important practical implications.
Commercial DBMSs contain a query optimizer module that converts queries
into expression involving relational operations and then uses these
mathematical properties to simplify those expressions and thus optimize
query execution.

\hypertarget{sql}{%
\subsection{SQL:}\label{sql}}

\hypertarget{basic-select-statement}{%
\subsubsection{Basic SELECT statement}\label{basic-select-statement}}

An application describes the access that it wants the DBMS to perform on
its behalf in a language supported by the DBMS. We are particularly
interested in SQL, the most commonly used database language, which
provides facilities for accessing a relational database and is supported
by almost all commercial DBMSs.

The basic structure of the SQL statements for manipulating data is
straightforward and easy to understand. Each statement takes one or more
tables as arguments and produces a table as a result. For example, to
find the name of the student whose Id is 987654321, we might use the
statement

\begin{lstlisting}[language=SQL, label=query1]
SELECT Name
FROM STUDENT 
WHERE Id = 987654321 
\end{lstlisting}

More precisely, this statement asks the DBMS to extract from the table
named in the FROM clause---that is, the table STUDENT---all rows
satisfying the condition in the WHERE clause---that is, all rows whose
Id column has value 987654321---and then from each such row to delete
all columns except those named in the SELECT clause---that is, Name. The
resulting rows are placed in a result table produced by the statement.
In this case, because Ids are unique, at most one row of STUDENT can
satisfy the condition, and so the result of the statement is a table
with one column and at most one row.

Thus, the FROM clause identifies the table to be used as input, the
WHERE clause identifies the rows of that table from which the answer is
to be generated, and the SELECT clause identifies the columns of those
rows that are to be output in the result table.

The result table generated by this example contains only one column and
at most one row. As a somewhat more complex example, the statement

\begin{lstlisting}[language=SQL]
SELECT Id, Name
FROM STUDENT 
WHERE Status = 'senior' 
\end{lstlisting}

returns a result table containing two columns and multiple rows:

\begin{longtable}[]{@{}ll@{}}
\caption{The database table returned by the SQL SELECT
statement}\tabularnewline
\toprule
Id & Name\tabularnewline
\midrule
\endfirsthead
\toprule
Id & Name\tabularnewline
\midrule
\endhead
987654321 & Bart Simpson\tabularnewline
023456789 & Homer Simpson\tabularnewline
\bottomrule
\end{longtable}

the Ids and names of all seniors. If we want to produce a table
containing all the columns of STUDENT but describing only seniors, we
use the statement

\begin{lstlisting}[language=SQL]
SELECT  *
FROM STUDENT
WHERE Status = 'senior'
\end{lstlisting}

The asterisk is simply shorthand that allows us to avoid listing the
names of all the columns of STUDENT. In some situations the user is
interested not in outputting a result table but in information about the
result table. An example is the statement

\begin{lstlisting}[language=SQL]
SELECT COUNT(*)
FROM STUDENT 
WHERE Status = 'senior' 
\end{lstlisting}

which returns the number of rows in the result table (i.e., the number
of seniors). COUNT is referred to as an aggregate function because it
produces a value that is a function of all the rows in the result table.
Note that in this case, the SELECT statement produces a table that has
only one row and one column.

The WHERE clause is the most interesting component of the SELECT
statement, it contains a general condition that is evaluated over each
row of the table named in the FROM clause. Column values from the row
are substituted into the condition, yielding an expression that has
either a true ora false value. If the condition evaluates to true, the
row is retained for processing by the SELECT clause and then stored in
the result table. Hence, the WHERE clause acts as a filter.

Conditions can be much more complex than we have seen so far: A
condition can be a Boolean combination of terms. If we want the result
table to contain information describing seniors whose Ids are in a
particular range, for example, we might use

\begin{lstlisting}[language=SQL]
WHERE Status = 'senior' AND Id > '888888888'
\end{lstlisting}

OR and NOT can also be used. Furthermore, a number of predicates are
provided in the language for expressing particular relationships. For
example, the IN predicate tests set membership.

\begin{lstlisting}[language=SQL]
WHERE Status IN ('freshman', ‘sophomore')
\end{lstlisting}

Additional aggregates and predicates and the full complexity of the
WHERE clause will be discussed separately in upcoming chapters.

\hypertarget{multi-table-select-statements}{%
\subsubsection{Multi-table SELECT
statements}\label{multi-table-select-statements}}

The result table can contain information extracted from several base
tables. Thus, if we have a table TRANSCRIPT with columns StudId,
CrsCode, Semester, and Grade, the statement

\begin{lstlisting}[language=SQL]
SELECT Name, CrsCode, Grade
FROM STUDENT, TRANSCRIPT
WHERE StudId = Id AND Status = 'senior' 
\end{lstlisting}

can be used to form a result table in which each row contains the name
of a senior, a particular course she took, and the grade she received.

The first thing to note is that the attribute values in the result table
come from different base tables: Name comes from STUDENT; CrsCode and
Grade come from TRANSCRIPT. As in the previous examples, the FROM clause
produces a table whose rows are input to the WHERE clause. In this case
the table is the Cartesian product of the tables listed in the FROM
clause: a row of this table is the concatenation of a row of STUDENT and
a row of TRANSCRIPT. Many of these rows make no sense. For example, Bart
Simpson's row in STUDENT is not related to a row in TRANSCRIPT
describing a course that Bart did not take. The first conjunct of the
WHERE clause ensures that the rows of TRANSCRIPT for a particular
student are associated with the appropriate row of STUDENT by matching
the Id values of the rows of the two tables. For example, if TRANSCRIPT
has a row \emph{(987654321, CS305, F1995, C)}, it will match only Bart
Simpson's row in STUDENT, producing the row \emph{(Bart Simpson, CS305,
C)} in the result table.

\hypertarget{query-optimization.}{%
\subsubsection{Query optimization.}\label{query-optimization.}}

One very important feature of SQL is that the programmer does not have
to specify the algorithm the DBMS should use to satisfy a particular
query.

For example, tables are frequently defined to include auxiliary data
structures, called indices, which make it possible to locate particular
rows without using lengthy searches through the entire table. Thus, an
index on the Id column of the STUDENT table might contain a list of
pairs (Id, pointer) where the pointer points to the row of the table
containing the corresponding Id. If such an index were present, the DBMS
would automatically use it to find the row that satisfies the query
{[}query1{]}. If the table also had an index on the column Status, the
DBMS would use that index to find the rows that satisfy the query
({[}sql2{]}). If this second index did not exist, the DBMS would
automatically use some other method to satisfy ({[}sql2{]})---for
example, it might look at every row in the table in order to locate all
rows having the value senior in the Status column. The programmer does
not specify what method to use---just the condition the desired result
table must satisfy.

In addition to selecting appropriate indices to use, the query optimizer
uses the properties of the relational operations to further improve the
efficiency with which a query can be processed---again, without any
intervention by the programmer. Nevertheless, programmers should have
some understanding of the strategies the DBMS uses to satisfy queries so
they can design the database tables, indices, and SQL statements in such
a way that they will be executed in an efficient manner consistent with
the requirements of the application.

\hypertarget{changing-the-contents-of-tables}{%
\subsubsection{Changing the contents of
tables}\label{changing-the-contents-of-tables}}

The following examples illustrate the SQL statements for modifying the
contents of a table. The statement

\begin{lstlisting}[language=SQL]
UPDATE STUDENT
SET Status = 'sophomore'
WHERE Id = '111111111' 
\end{lstlisting}

updates the STUDENT table to make \emph{John Doe} a \emph{sophomore}.
The statement

\begin{lstlisting}[language=SQL]
INSERT 
INTO STUDENT (Id, Name, Address, Status) 
VALUES ('999999999', 'Winston Churchill', '10 Downing St', 'senior') 
\end{lstlisting}

inserts a new row for \emph{Winston Churchill} in the \emph{STUDENT}
table.

The statement

\begin{lstlisting}[language=SQL]
DELETE 
FROM STUDENT 
WHERE Id = '1111111111'
\end{lstlisting}

deletes the row for \emph{John Doe} from the \emph{STUDENT} table.
Again, the details of how these operations are to be performed need not
be specified by the programmer.

\hypertarget{creating-tables-and-specifying-constraints}{%
\subsubsection{Creating tables and specifying
constraints}\label{creating-tables-and-specifying-constraints}}

Before you can store data in a table, the table structure must be
created. For instance, the STUDENT table could have been created with
the SQL statement

\begin{lstlisting}[language=SQL]
CREATE TABLE STUDENT (
 Id INTEGER, 
 Name CHAR(20), 
 Address CHAR(50), 
 Status CHAR(10), 
 PRIMARY KEY (Id) ) 
\end{lstlisting}

where we have declared the name of each column and the domain (type) of
the data that can be stored in that column. We have also declared the Id
column to be a primary key to the table, which means that each row of
the table must have a unique value in that column and the DBMS will
(most probably) automatically construct an index on that column. The
DBMS will enforce this uniqueness constraint by not allowing any
\emph{INSERT} or \emph{UPDATE} statement to produce a row with a value
in the \emph{Id} column that duplicates a value of Id in another row.
This requirement is an example of an integrity constraint (sometimes
called a consistency constraint)an application-based restriction on the
values that can appear as entries in the database.

We have given simple examples of each statement type to highlight the
conceptual simplicity of the basic ideas underlying SQL, but be aware
that the complete language has many subtleties. Each statement type has
a large number of options that allow very complex queries and updates.
For this reason, mastery of SQL requires significant effort.

\end{document}
